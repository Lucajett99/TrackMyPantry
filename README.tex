Introduzione {#introduzione .unnumbered}
============

Track my Pantry è un’applicazione sviluppata per il corso di Laboratorio
di applicazioni mobili per il C.d.L. di Informatica (a.a. 2020/21).\
L’ obiettivo dell’ applicazione è tenere traccia dei prodotti in
dispensa attraverso il barcode associato al prodotto, interrogando un
Web Service (database condiviso) dove è possibile aggiungere e
richiedere i dettagli dei prodotti. In questo report descrivo le
principali caratteristiche e le tecniche implementative utilizzate per
lo sviluppo.

Dettagli implementativi
=======================

L’ applicazione è stata sviluppata per Android in modo ibrido
utilizzando Ionic, Capacitor e il framework React.\
Le chiamate al Web Service sono create mediante la libreria Fetch, per
il database è stata utilizzato il plugin SQLite di capacitor e per
l’accesso alla fotocamera è stato utilizzato il plugin Camera sempre di
capacitor.\
Il minSdk dell’applicazione è il 23, in quanto il plugin SQLite è
supportato dal 23 in poi.\
Le pagine dell’app sono quattro:

-   Login

-   Registration

-   Home

-   Shopping List

L’app si sviluppa fondamentalmente intorno alle due pagine ’Home’ e
’Shopping List’, le altre due ’Login’ e ’Registration’ vengono
,ovviamente, utilizzate solo ai fini della login e della registrazione.\
Sono stati utilizzati solamente componenti Ionic, in modo da rendere
l’applicazione il più flessibile possibile.\
Ho utilizzato la routing di Ionic con una tab bar a tre tab. La prima è
la home, la seconda è la shopping list e la terza non è una vera e
propria tab ma serve semplicemente per fare il logout.\
La pagina di avvio dell’app cambia in base a se l’utente è stato già
autenticato o meno (dopo 7 giorni dalla login, l’ access token scade),
se è autenticato la pagina base sarà /Home altrimenti sarà /Login.\

Funzionalità
============

Funzionalità di base
--------------------

I requisiti minimi dell’app sono stati tutti sviluppati e sono i
seguenti:

-   Aggiornare il database condiviso

-   Effettuare il login e registrarsi per accedere al database remoto

-   Tienere traccia della spesa acquistata

Funzionalità aggiuntive
-----------------------

Qui di seguito, invece, troviamo le funzionalità da me aggiunte:

-   Aumento e diminuzione della quantità di un prodotto

-   Barcode scanner

-   Filtrare i prodotti locali per nome

-   Foto per il caricamento di un prodotto

-   Lista della spesa (Shopping list)

-   Immagini dei prodotti

-   Gestione degli utenti diversi

Login e Registrazione
=====================

Dalle immagini seguenti si può vedere che sono due pagine molto semplici
e molto minimali

<span>2</span>

![image](Login) ![image](Registration)

Login
-----

Questa pagina contiene due campi:

1.  E-mail

2.  Password

Ovviamente finché non verrà eseguito l’accesso, la tab bar rimarrà
disabilitata. In caso di successo del login, l’app porterà l’utente
nella pagina principale “Home” e abiliterà la tab bar.

### Autenticazione

Per quanto riguarda l’autenticazione, essa viene gestita memorizzando
l’access token al momento del login.\
Ho utilizzando la libreria “Storage” che mi permette di usufruire delle
funzioni get e set per restituire e settare dei valori semplici in
memoria. Come si può vedere dall’immagine (righe 14-16), memorizzo
l’email, l’access token e la data corrente. La data la utilizzo per
controllare la validità dell’access token all’apertura dell’app, dato
che dopo 7 giorni scade.

    export async function login(email: string, password: string) {
        const loginData = {
            email: email,
            password: password
        };
        const requestOptions = {
            method: 'POST',
            headers: { 'Content-Type': 'application/json' },
            body: JSON.stringify(loginData)
        };
        const response = await fetch(baseURL + '/auth/login', requestOptions);
        if(response.ok) {
            const token = await response.clone().json();
            await setValue("accessToken", token);
            await setValue("email", email);
            await setValue("date", new Date());
        }
        return response;
    }

     export const isAuthed = async () =>{
      const token = await getValue("accessToken");
      const date = new Date(await getValue("date"));
      const today = new Date();
      date.setDate(date.getDate() + 7);
      if(token != null && date.getTime() > today.getTime())
        return true;
      else
        return false;
    }

Registrazione
-------------

Questa pagina è molto simile alla precedente. Ha un campo in più per lo
username e nel caso la registrazione avvenga con successo l’app porterà
l’utente sulla pagina di Login.

Home
====

Questa è la sezione principale dell’app, dove c’è una searchbar (filtra
i prodotti per nome), un bottone per aggiungere un nuovo prodotto e una
IonList che a sua volta contiente tutti i prodotti salvati localmente in
un database.\
La searchbar lavora solo su una copia salvata dei prodotti, quindi non
necessita di eseguire ogni volta una query al database locale.

Card
----

Ogni prodotto viene rappresentato da una card, in cui è presente:

<span>2</span>

-   Il nome del prodotto

-   La quantità disponibile

-   Due bottoni: uno per aumentare e uno per diminuire la quantità

-   Un bottone per eliminare definitivamente il prodotto dal database
    locale

-   Un bottone per altri dettagli del prodotto (finestra con descrizione
    e barcode)

<span>.5</span> ![image](Home)

<span>.5</span> ![image](Home_detail)

La particolarità dell’app sta nel fatto che se la quantità dovesse
divenire uguale a 0, la card verrà automaticamente rimossa dalla Home e
inserita nella Shopping list. Nel caso venga nuovamente comprato (quindi
cercato) quel prodotto, che in quel momento si trova nella Shopping
list, verrà reinserito nella Home con quantità = 1 ed eliminato dalla
Shopping list.\
L’idea è quella di dare la possibilità all’utente di eliminare
definitivamente un prodotto dalla dispensa, nel caso in cui non se ne
faccia più uso, ma anche di mantenerlo se è un prodotto che l’utente
compra spesso ma lo ha esaurito.

Ricerca del prodotto
--------------------

Nel caso si volesse aggiungere un nuovo prodotto alla dispensa, il
bottone con l’icona + farà aprire una finestra modale, dove ci sarà la
possibilità di cercare nel database condiviso un prodotto tramite
barcode (sia stringa che con scannerizzazione del barcode).\
Il contenuto della finestra è uno stato (utilizzando useState di React)
che viene cambiato in base al contesto (scanner, risultati prodotti,
creazione di un nuovo prodotto).\
Il risultato della ricerca saranno sempre delle card simili alle
precedenti con tutti i dettagli del prodotto visualizzabili cliccando il
bottone di informazione. A questo punto basta cliccare “ADD” e verrà
automaticamente aggiunto il prodotto selezionato con quantità = 1, però
solo dopo aver dato un rating da 1 a 5 (partirà una chiamata HTTP per
dare il voto al prodotto).\
Per la scannerizzazione del barcode ho utilizzato un plugin di Capacitor
’barcode-scanner’.

<span>.5</span> ![image](Barcode)

<span>.5</span> ![image](Product_list)

Aggiunta di un nuovo prodotto
-----------------------------

<span>2</span>

Se l’utente non è convinto dei prodotti già presenti o il prodotto
(barcode) non è proprio presente nel database remoto, c’è la possibilità
di crearne uno.\
A quel punto si rimarrà sempre sulla stessa finestra modale ma cambierà
il contenuto con la possibilità di creare un nuovo prodotto con i
seguenti dettagli:

-   Foto scattata in quel momento (opzionale)

-   Nome del prodotto

-   Descrizione del prodotto

Dopo aver creato il prodotto, esso verrà automaticamente aggiunto alla
dispensa con quantità = 1.

![image](NewProduct)

Shopping List
=============

<span>2</span>

In questa sezione dell’app ci sono tutti i prodotti del database locale
che hanno quantità = 0.\
In questo modo riesco a separare per bene i prodotti che ho nella
dispensa, con quelli che utilizzo di solito ma sono esauriti.\
Per aggiornarla di volta in volta utilizzo un evento che viene chiamato
ogni qual volta un prodotto raggiunge quantità = 0 o quando viene
nuovamente comprato (quindi ricercato nel database condiviso).

![image](ShoppingList)

Gestione utenti diversi
=======================

E’ possibile quindi utilizzare l’applicazione con più utenti diversi ed
i prodotti verranno correttamente separati fra questi grazie alla
procedura di Login.\
Questo perché per ogni prodotto memorizzato nel database locale avrà
anche il campo e-mail, in modo tale che ogni volta che voglio ottenere i
prodotti prenderò solo e soltanto quelli dell’utente corrente, quindi
quelli con l’e-mail che combacierà nella query.\
Infatti ogni volta che l’app esegue una query al database locale,
utilizzerà anche l’e-mail dell’utente corrente.
